% \iffalse
\let\negmedspace\undefined
\let\negthickspace\undefined
\documentclass[journal,12pt,twocolumn]{IEEEtran}
\usepackage{cite}
\usepackage{amsmath,enumitem,amssymb,amsfonts,amsthm}
\usepackage{algorithmic}
\usepackage{graphicx}
\usepackage{float}
\usepackage{textcomp}
\usepackage{xcolor}
\usepackage{caption}
\usepackage{txfonts}
\usepackage{listings}
\usepackage{enumitem}
\usepackage{mathtools}
\usepackage{gensymb}
\usepackage{comment}
\usepackage[breaklinks=true]{hyperref}
\usepackage{tkz-euclide} 
\usepackage{listings}
\usepackage{tabularx}
\usepackage{gvv}                                        
\def\inputGnumericTable{}                                 
\usepackage[latin1]{inputenc}                              
\usepackage{color}                                            
\usepackage{array}                                            
\usepackage{longtable}                                       
\usepackage{calc}                                             
\usepackage{multirow}                                         
\usepackage{hhline}                                           
\usepackage{ifthen}                                        
\usepackage{lscape}
\newtheorem{theorem}{Theorem}[section]
\newtheorem{problem}{Problem}
\newtheorem{proposition}{Proposition}[section]
\newtheorem{lemma}{Lemma}[section]
\newtheorem{corollary}[theorem]{Corollary}
\newtheorem{example}{Example}[section]
\newtheorem{definition}[problem]{Definition}
\newcommand{\BEQA}{\begin{eqnarray}}
\newcommand{\EEQA}{\end{eqnarray}}
\newcommand{\define}{\stackrel{\triangle}{=}}
\theoremstyle{remark}
\newtheorem{rem}{Remark}
\usepackage{float}
\usepackage{adjustbox}


\begin{document}
\bibliographystyle{IEEEtran}
\vspace{3cm}


\title{Exempler- 11.9.2.12}
\author{EE22BTECH11005- Ambati Krishna Kaustubh$^{*}$% <-this % stops a space
}

\maketitle
\newpage
\bigskip



\textbf{Question:}

The ratio of sums of m and n terms of an A.P. is $m^2:n^2$.Show
that the ratio of $m^{th}$ and $n^{th}$ term is (2m-1):(2n-1).

{\textbf{Solution:}}



nth term of an A.P. is given by 
\begin{align}y(n)=y(0)+nd\end{align}
sum of n terms is given by
\begin{align}
    x(n)=\sum_{k=0}^n y(k)
\end{align}
\begin{align}x(n)=\frac{n+1}{2}(2x(0)+nd)\end{align}

Given,
\begin{align}\frac{x(m)}{x(n)}=\frac{m^2}{n^2} \label{eq:sum} \end{align} 
now substituting x(m) and x(n) into the (4) we get,
\begin{align}\frac{\frac{m+1}{2} {(2x(0)+md)}}{\frac{n+1}{2} {(2x(0)+nd)}}&= 
 \frac{m^2}{n^2}\end{align}
 \begin{align}\frac{nm^2+m^2}{2x(0)+nd}&=\frac{n^2m+n^2}{2x(0)+md}
 \end{align}
 \begin{align}
     2x(0)nm^2+2x(0)m^2+m^2nd
     =2x(0)n^2m+n^2md+2 x(0)n^2
 \end{align}
\begin{align} 2x(0)(m^2-n^2+nm^2-mn^2)=d(n^2m-m^2n)\end{align}
\begin{align}
    2x(0)(m+n+mn)=-d(mn)
\end{align}
\begin{align}
     2x(0)=\frac{-mnd}{m+n+mn}
\end{align}
The ratio between $m^{th}$ and $n^{th}$ terms is given by
 \begin{align}\frac{y(m)}{y(n)}=\frac{y(0)+md}{y(0)+nd}\end{align}
 
\begin{align} \frac{y(m)}{y(n)}=\frac{m(2m+n+2mn)}{n(2n+m+2mn)}\end{align}
\pagebreak


\begin{table}[H]
    \center
    \renewcommand\thetable{1}
 

\def\arraystretch{3}
\begin{adjustbox}{width=0.5\textwidth}
    \begin{tabular}{|c|c|c|}
    \hline
        \textbf{Parameter}&\textbf{Description}&\textbf{Value}\\
        \hline
        n&an integer&1,2,3...\\
        \hline
        $y(n)$& general term of an A.P.&$(y(0)+nd)\cdot{u(n)}$ \\
        \hline
        $x\brak{n}$ &sum of n terms of an A.P.&$ \frac{n+1}{2}(2x(0)+nd)\cdot{u(n)}$ \\
        \hline
        $x(m):x(n)$&ratio of mth term is to nth term&$m^2:n^2$ \\
      \hline
       \end{tabular} 
            \end{adjustbox}
    \caption{Parameter Table}

    \label{tab:11.9.2.12.1}
\end{table}

\begin{enumerate}
 \item Analysis of equation for sum of n terms of an A.P:\\[4pt]
   
 By the differentiation property :

\begin{align}
   n x\brak{n} &\xleftrightarrow{\mathcal{Z}} \brak{-z} \frac{dX\brak{z}}{dz} \\
\implies  n u\brak{n} &\xleftrightarrow{\mathcal{Z}} \frac{z^{-1}}{\brak{1-z^{-1}}^2} ,\cbrak{z\in\mathbb{C} : \lvert z \rvert > 1} \\
\implies     n^2 u\brak{n}&\xleftrightarrow{\mathcal{Z}}\frac{z^{-1}\brak{z^{-1}+1}}{\brak{1-z^{-1}}^3} ,\cbrak{z\in\mathbb{C} : \lvert z \rvert > 1}\end{align}


\begin{align}
   n^2\cdot u(n)\xleftrightarrow{\mathcal{Z}} \frac{z^{-1}(1+z^{-1})}{(1-z^{-1})^{3}}, {\{\hspace{4pt} {z\in\mathbb{C} : \lvert z \rvert >1}\}} \end{align} \label{eq:diff2}
\begin{align} X\brak{z} &= \sum_{n=-\infty}^\infty x\brak{n}\cdot z^{-n}\end{align} \\
\begin{align} X\brak{z} &= \sum_{n=0}^\infty \frac{n+1}{2}(2x(0)+nd)\cdot{u(n)}\cdot z^{-n}\end{align} \\
\begin{align}
   X(z)=2x(0)U(z)-2x(0)z\frac{dU(z}{dz}+dz^2\frac{d^{2}U(z)}{dz^2}
\end{align}

\begin{align}
 \therefore  X(z)=\frac{z^{-1}(d-x(0))+x(0)}{2(({1-z^{-1}})^{3})} ,\{\hspace{4pt} {z\in\mathbb{C} : \lvert z \rvert >1\}} \end{align}
\item 
Analysis equation for the $n^{th}$term of an A.P.:\\
\begin{align}
    y(n)=y(0)+nd
\end{align}
\begin{align}
    n\cdot u(n)&\xleftrightarrow{\mathcal{Z}}\frac{z^-1}{\brak{{1-z^{-1}}}^{-2}}, {\{\hspace{4pt} {z\in\mathbb{C} : \lvert z \rvert >1}\}}
\end{align}
\begin{align} Y\brak{z} &= \sum_{n=-\infty}^\infty y\brak{n}\cdot z^{-n}\end{align} \\
\begin{align} Y\brak{z} &= \sum_{n=0}^\infty (y(0)+nd)\cdot{u(n)}\cdot z^{-n}\end{align} \\
\begin{align}
    Y(z)=y(0)U(z)-zd\frac{dU(z)}{dz}
\end{align}
\begin{align}
   \therefore Y\brak{z}= \frac{y(0)+\brak{d-y(0)}z^{-1}}{(1-z^{-1})^{2}}, \{\hspace{4pt} {z\in\mathbb{C} : \lvert z \rvert >1\}}
\end{align}
 \end{enumerate}




\renewcommand{\thefigure}{\theenumi}
\renewcommand{\thetable}{\theenumi}
\end{document}
