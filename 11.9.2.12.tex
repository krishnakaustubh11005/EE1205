% \iffalse
\let\negmedspace\undefined
\let\negthickspace\undefined
\documentclass[journal,12pt,twocolumn]{IEEEtran}
\usepackage{cite}
\usepackage{amsmath,amssymb,amsfonts,amsthm}
\usepackage{algorithmic}
\usepackage{graphicx}
\usepackage{textcomp}
\usepackage{xcolor}
\usepackage{txfonts}
\usepackage{listings}
\usepackage{enumitem}
\usepackage{mathtools}
\usepackage{gensymb}
\usepackage{comment}
\usepackage[breaklinks=true]{hyperref}
\usepackage{tkz-euclide} 
\usepackage{listings}
\usepackage{gvv}                                        
\def\inputGnumericTable{}                                 
\usepackage[latin1]{inputenc}                                
\usepackage{color}                                            
\usepackage{array}                                            
\usepackage{longtable}                                       
\usepackage{calc}                                             
\usepackage{multirow}                                         
\usepackage{hhline}                                           
\usepackage{ifthen}                                           
\usepackage{lscape}

\newtheorem{theorem}{Theorem}[section]
\newtheorem{problem}{Problem}
\newtheorem{proposition}{Proposition}[section]
\newtheorem{lemma}{Lemma}[section]
\newtheorem{corollary}[theorem]{Corollary}
\newtheorem{example}{Example}[section]
\newtheorem{definition}[problem]{Definition}
\newcommand{\BEQA}{\begin{eqnarray}}
\newcommand{\EEQA}{\end{eqnarray}}
\newcommand{\define}{\stackrel{\triangle}{=}}
\theoremstyle{remark}
\newtheorem{rem}{Remark}
\begin{document}

\bibliographystyle{IEEEtran}
\vspace{3cm}

\title{Exempler- 11.9.2.12}
\author{EE22BTECH11005- Ambati Krishna Kaustubh}% <-this % stops a space

\maketitle
\newpage
\bigskip


\parindent 0px
\title{\textbf{Exempler-11.9.2.12}
}
\author{\textbf{\begin{arge}EE23BTECH11005\end{arge}}-Ambati Krishna Kaustubh}

\maketitle

\textbf{Question:}

The ratio of sums of m and n terms of an A.P. is $m^2:n^2$.Show

that the ratio of $m^{th}$ and $n^{th}$ term is (2m-1):(2n-1).

{\textbf{Solution:}}


The general equation for the sum of n terms of an A.P. is given by
\begin{align}S(n)=\frac{n}{2} {(2a+(n-1)d)}\end{align}
The general equation for the $n^{th }$ term of an A.P is given by
\begin{align}T(n)=a+(n-1)d\end{align}
where a is the first term and d is the common difference of A.P.\\[4pt]

Given,
\begin{align}\frac{S(n)}{S(m)}=\frac{m^2}{n^2}\end{align}
now substituting S(m) and S(n) into the (2) we get,
\begin{align}\frac{\frac{n}{2} {(2a+(n-1)d)}}{\frac{m}{2} {(2a+(n-1)d)}}= 
 {\frac{m^2}{n^2}}\end{align}
 By cross multiplying and solving we get a relation between a and d that is
 \begin{align}2a=d\end{align}
 The ratio between $m^{th}$ and $n^{th}$ terms is given by
 \begin{align}\frac{T(m)}{T(n)}=\frac{a+(m-1)d}{a+(n-1)d}\end{align}
 
 Now substituting (5) in (6) we get \textbf{T(m):T(n)=(2m-1):(2n-1)}.



\renewcommand{\thefigure}{\theenumi}
\renewcommand{\thetable}{\theenumi}
\end{document}
