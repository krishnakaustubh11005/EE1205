% \iffalse
\let\negmedspace\undefined
\let\negthickspace\undefined
\documentclass[journal,12pt,twocolumn]{IEEEtran}
\usepackage{cite}
\usepackage{amsmath,amssymb,amsfonts,amsthm}
\usepackage{algorithmic}
\usepackage{graphicx}
\usepackage{textcomp}
\usepackage{xcolor}
\usepackage{txfonts}
\usepackage{listings}
\usepackage{enumitem}
\usepackage{mathtools}
\usepackage{gensymb}
\usepackage{comment}
\usepackage[breaklinks=true]{hyperref}
\usepackage{tkz-euclide} 
\usepackage{listings}
\usepackage{gvv}                                        
\def\inputGnumericTable{}                                 
\usepackage[latin1]{inputenc}                                
\usepackage{color}                                            
\usepackage{array}                                            
\usepackage{longtable}                                       
\usepackage{calc}                                             
\usepackage{multirow}                                         
\usepackage{hhline}                                           
\usepackage{ifthen}                                           
\usepackage{lscape}

\newtheorem{theorem}{Theorem}[section]
\newtheorem{problem}{Problem}
\newtheorem{proposition}{Proposition}[section]
\newtheorem{lemma}{Lemma}[section]
\newtheorem{corollary}[theorem]{Corollary}
\newtheorem{example}{Example}[section]
\newtheorem{definition}[problem]{Definition}
\newcommand{\BEQA}{\begin{eqnarray}}
\newcommand{\EEQA}{\end{eqnarray}}
\newcommand{\define}{\stackrel{\triangle}{=}}
\theoremstyle{remark}
\newtheorem{rem}{Remark}
\begin{document}

\bibliographystyle{IEEEtran}
\vspace{3cm}

\title{Exempler- 11.9.2.12}
\author{EE22BTECH11005- Ambati Krishna Kaustubh}% <-this % stops a space

\maketitle
\newpage
\bigskip


\parindent 0px
\title{\textbf{Exempler-11.9.2.12}
}
\author{\textbf{\begin{arge}EE23BTECH11005\end{arge}}-Ambati Krishna Kaustubh}

\maketitle

\textbf{Question:}

The ratio of sums of m and n terms of an A.P. is $m^2:n^2$.Show
that the ratio of $m^{th}$ and $n^{th}$ term is (2m-1):(2n-1).

{\textbf{Solution:}}


The general equation for the sum of n terms of an A.P. is given by
\begin{align}S(n)=\frac{n}{2} {(2a+(n-1)d)}\end{align}
The general equation for the $n^{th }$ term of an A.P is given by
\begin{align}T(n)=a+(n-1)d\end{align}
where a is the first term and d is the common difference of A.P.\\[4pt]

Given,
\begin{align}\frac{S(m)}{S(n)}=\frac{m^2}{n^2}\end{align}
now substituting S(m) and S(n) into the (2) we get,
\begin{align}\frac{\frac{m}{2} {(2a+(m-1)d)}}{\frac{n}{2} {(2a+(n-1)d)}}= 
 {\frac{m^2}{n^2}}\end{align}
 By cross multiplying we get 
 \begin{align}\frac{n}{2a+(m-1)d}=\frac{m}{2a+(n-1)d}\end{align}
 solving the above equation we get
 \begin{align}2a=d\end{align}
 The ratio between $m^{th}$ and $n^{th}$ terms is given by
 \begin{align}\frac{T(m)}{T(n)}=\frac{a+(m-1)d}{a+(n-1)d}\end{align}
 
 Now substituting (5) in (6) we get \begin{align}\frac{T(m)}{T(n)}=\frac{a+(m-1)2a}{a+(n-1)2a}\end{align}

 
 
By cancelling a in both the denominator and numerator we get \textbf{T(m):T(n)=(2m-1):(2n-1)}.
if we consider that the A.P is starting from n=0 we will have sum of n terms and nth term as 
\begin{align}S(n)=\frac{n+1}{2}(2a+nd)\end{align}
\begin{align}x(n)=\frac{n+1}{2}(2x(0)+nd)\end{align}
\begin{align}T(n)=y(0)+nd\end{align}
\begin{align}y(n)=y(0)+nd\end{align}
Given,
\begin{align}\frac{x(m)}{x(n)}=\frac{m^2}{n^2}\end{align}
now substituting x(m) and x(n) into the (9) we get,
\begin{align}\frac{\frac{m+1}{2} {(2x(0)+md)}}{\frac{n+1}{2} {(2x(0)+nd)}}&= 
 \frac{m^2}{n^2}\end{align}
 \begin{align}\frac{nm^2+m^2}{2x(0)+nd}&=\frac{n^2m+n^2}{2x(0)+md}
 \end{align}
 \begin{align}
     2x(0)nm^2+2x(0)m^2+m^2nd
     =2x(0)n^2m+n^2md+2 x(0)n^2
 \end{align}
\begin{align} 2x(0)(m^2-n^2+nm^2-mn^2)=d(n^2m-m^2n)\end{align}
\begin{align}
    2x(0)(m+n+mn)=-d(mn)
\end{align}
\begin{align}
     2x(0)=\frac{-mnd}{m+n+mn}
\end{align}
The ratio between $m^{th}$ and $n^{th}$ terms is given by
 \begin{align}\frac{y(m)}{y(n)}=\frac{y(0)+md}{y(0)+nd}\end{align}
 Now substituting y(0)(same as x(0))in (20) we get 
\begin{align} \frac{y(m)}{y(n)}=\frac{m(2m+n+2mn)}{n(2n+m+2mn)}\end{align}
\pagebreak


\renewcommand{\thefigure}{\theenumi}
\renewcommand{\thetable}{\theenumi}

\end{document}
